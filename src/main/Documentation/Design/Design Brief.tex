% !TEX TS-program = pdflatex
% !TEX encoding = UTF-8 Unicode

% This is a simple template for a LaTeX document using the "article" class.
% See "book", "report", "letter" for other types of document.

\documentclass[11pt]{article} % use larger type; default would be 10pt


%%% The "real" document content comes below...

\title{Design Brief}
\author{Daniel Hopkins}
\date{16/09/2014} % Activate to display a given date or no date (if empty),
         % otherwise the current date is printed  

\begin{document}

\maketitle
\tableofcontents

\section{Introduction}

In this document I will be discussing the specification provided for the project and looking at what requirements I can get from the given brief. To do this I will be analysing the entire brief and breaking it down into goals. This will be useful as it will allow me to figure out exactly what is required from the project, as well as giving a solid foundation from which to build on.

Firstly I will be looking at the actual brief. Then I will break it down into a set of requirements. Lastly I will be thinking of any extra things I would like to add to the project as possible extra features.

\section{The Brief}
\subsection{The given Brief}
A ball bounces around in a pinball machine. This is a typical situation in which a small difference in initial conditions makes a vast difference as a system evolves. This project will animate the bouncing ball and look at how minor variations in initial conditions cause instability and fractal behaviour in the dynamics, and also how varying the accuracy of floating point representation affects the apparent behaviour of the solution.

\subsection{Comments}

This brief gives the idea that the project will require the making of a virtual pinball machine, the animation of the ball and paddles to control the flow of the ball itself, and possibly tracking the movement of the ball as it flows around the system. In order to do this I will need to learn how to perform collision detection and similar things as well as how to program animation.

To this end I will be spending a lot of the early portion of this project researching new technologies in order to understand exactly what I am doing. This will be done using a number of free online resources to learn how to do these things.

\end{document}
